\documentclass{article}
\usepackage[utf8]{inputenc}
\usepackage{lipsum} % Para generar texto de prueba

\title{Arquitectura Móvil}
\author{}
\date{}

\begin{document}

\maketitle

La arquitectura móvil se refiere a cómo se construyen y organizan los elementos de diseño que conforman una aplicación móvil. Describe los patrones y técnicas utilizados para diseñar y desarrollar estas aplicaciones, proporcionando un plan y prácticas recomendadas para crear una aplicación bien estructurada.

En términos de arquitectura de aplicaciones, hay dos tipos principales: frontend y backend. El frontend se ocupa de la experiencia del usuario, mientras que el backend maneja el acceso a los datos, los servicios y otros sistemas necesarios para que la aplicación funcione correctamente.

Tipos de arquitectura de aplicaciones:

\begin{itemize}
    \item \textbf{Arquitectura cliente-servidor}: Es la más común en aplicaciones web. Se divide en una parte cliente, que incluye la interfaz de usuario o el código frontend, y una parte servidor, que se encarga del almacenamiento de datos, la lógica de negocio y la comunicación con servicios externos.
    
    \item \textbf{Arquitectura de una sola página (Single Page Application - SPA)}: Este enfoque popular almacena todo el código necesario para ejecutar la aplicación en una sola página accesible desde cualquier dispositivo con conexión a Internet. Utiliza JavaScript y HTML5 para crear interfaces de usuario dinámicas, rápidas y receptivas.
\end{itemize}

\end{document}