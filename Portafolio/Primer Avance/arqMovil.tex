\documentclass{article}
\usepackage[utf8]{inputenc}
\usepackage{lipsum} % For generating dummy text

\title{Mobile Architecture}
\author{}
\date{}

\begin{document}

\maketitle

Mobile architecture refers to how the design elements that make up a mobile application are constructed and organized. It describes the patterns and techniques used to design and develop these applications, providing a plan and best practices for creating a well-structured app.

In terms of application architecture, there are two main types: frontend and backend. The frontend deals with the user experience, while the backend handles access to data, services, and other systems necessary for the application to function correctly.

Types of application architecture:

\begin{itemize}
    \item \textbf{Client-server architecture}: This is the most common in web applications. It is divided into a client part, which includes the user interface or frontend code, and a server part, which handles data storage, business logic, and communication with external services.
    
    \item \textbf{Single Page Application (SPA) architecture}: This popular approach stores all the code needed to run the application on a single page accessible from any device with an internet connection. It uses JavaScript and HTML5 to create dynamic, fast, and responsive user interfaces.
\end{itemize}

\end{document}
