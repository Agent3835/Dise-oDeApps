\documentclass{article}
\usepackage[utf8]{inputenc}
\usepackage{lipsum} % Para generar texto de prueba

\title{Patrones de Diseño para Aplicaciones Móviles}
\author{}
\date{}

\begin{document}

\maketitle

Los principios del diseño móvil tienen sus pilares cimentados en algunas características que las aplicaciones deben cumplir. Algunas interfieren directamente con la experiencia del usuario, ya que el diseño habla específicamente de lo que el usuario puede apreciar.

La usabilidad de una aplicación se enfoca principalmente en las siguientes características:

\begin{itemize}
    \item \textbf{Simplicidad:} La simplicidad está directamente relacionada con la usabilidad. Mantener algo simple significa que habrá pocos elementos, lo que facilita la navegación e interacción intuitiva. Las funciones dentro de la simplicidad estimulan la limpieza y la mínima complejidad.
    
    \item \textbf{Consistencia:} El aspecto del sistema debe coincidir con el resto de las ventanas, ya que representa un estilo que debe ser respetado y repetido en todo el software.
    
    \item \textbf{Navegación intuitiva:} La navegación intuitiva se vincula con la consistencia. Cada sistema operativo presenta diversos elementos, como botones, pestañas y paneles, para facilitar la navegación dentro de la aplicación. Utilizar estos elementos contribuirá a que el usuario los identifique de inmediato y, solo con la presencia de estos componentes, sea capaz de comprender cómo desplazarse de una sección a otra.
\end{itemize}

\end{document}
