\documentclass{article}
\usepackage[utf8]{inputenc}
\usepackage{lipsum} % For generating dummy text

\title{Design Patterns for Mobile Applications}
\author{}
\date{}

\begin{document}

\maketitle

The principles of mobile design have their foundations rooted in certain characteristics that applications must fulfill. Some directly affect the user experience, as design specifically speaks to what the user can appreciate.

The usability of an application focuses mainly on the following characteristics:

\begin{itemize}
    \item \textbf{Simplicity:} Simplicity is directly related to usability. Keeping something simple means there will be few elements, making navigation and interaction intuitive. Functions within simplicity encourage cleanliness and minimal complexity.
    
    \item \textbf{Consistency:} The look of the system should match the rest of the windows, as it represents a style that must be respected and repeated throughout the software.
    
    \item \textbf{Intuitive Navigation:} Intuitive navigation is linked with consistency. Each operating system presents various elements such as buttons, tabs, and panels to facilitate navigation within the application. Using these elements will help the user identify them immediately and, simply with the presence of these components, be able to understand how to move from one section to another.
\end{itemize}

\end{document}
