\documentclass{article}
\usepackage[utf8]{inputenc}
\usepackage{lipsum} % Para generar texto de prueba

\title{Aplicaciones Nativas y No Nativas}
\author{}
\date{}

\begin{document}

\maketitle

\textbf{Nativas:}

Las aplicaciones nativas son aquellas diseñadas específicamente para operar dentro de un solo sistema operativo, lo que significa que están optimizadas para funcionar en ese entorno y no pueden ejecutarse en otros sistemas. Estas aplicaciones aprovechan las características únicas del sistema operativo en el que se desarrollaron, lo que a menudo resulta en un rendimiento superior y una integración más fluida con el dispositivo.

\textbf{No Nativas:}

Por otro lado, las aplicaciones no nativas, también conocidas como aplicaciones multiplataforma o basadas en la web, son aquellas que pueden ejecutarse en varios sistemas operativos sin necesidad de modificaciones extensas. Estas aplicaciones suelen desarrollarse utilizando tecnologías web estándar como HTML, CSS y JavaScript, lo que les permite ser accesibles desde una variedad de dispositivos y plataformas, incluidos dispositivos móviles, computadoras de escritorio y navegadores web.

A pesar de que las aplicaciones nativas ofrecen un rendimiento óptimo y funcionalidades adaptadas al sistema operativo específico, las aplicaciones no nativas ofrecen una mayor flexibilidad y portabilidad al funcionar en múltiples plataformas con un solo conjunto de código. La decisión entre una aplicación nativa o no nativa depende de los requisitos del proyecto, las necesidades del usuario y las consideraciones de desarrollo específicas.

\end{document}
