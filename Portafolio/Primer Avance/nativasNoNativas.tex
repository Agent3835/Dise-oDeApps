\documentclass{article}
\usepackage[utf8]{inputenc}
\usepackage{lipsum} % For generating dummy text

\title{Native and Non-Native Applications}
\author{}
\date{}

\begin{document}

\maketitle

\textbf{Native:}

Native applications are those designed specifically to operate within a single operating system, meaning they are optimized to work in that environment and cannot run on other systems. These applications leverage the unique features of the operating system they were developed for, often resulting in superior performance and smoother integration with the device.

\textbf{Non-Native:}

On the other hand, non-native applications, also known as cross-platform or web-based applications, are those that can run on multiple operating systems without extensive modifications. These applications are typically developed using standard web technologies such as HTML, CSS, and JavaScript, allowing them to be accessible from a variety of devices and platforms, including mobile devices, desktop computers, and web browsers.

While native applications offer optimal performance and functionalities tailored to the specific operating system, non-native applications offer greater flexibility and portability by functioning on multiple platforms with a single codebase. The decision between a native or non-native application depends on the project requirements, user needs, and specific development considerations.

\end{document}
